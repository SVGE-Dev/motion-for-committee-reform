\documentclass[10pt, a4paper]{article}

\usepackage{graphicx}
\usepackage{fullpage}
\usepackage{url}
\usepackage{parskip}

\title{Proposals: Restructure of the Committee of SVGE and Reformation of the Disciplinary Process of SVGE}
\author{Frederick Fillingham, William Thomas}

\begin{document}

\maketitle

\section{Introduction:}
The Southampton Video Games and Esports Society (SVGE) currently has 19 committee roles, all of these roles are deemed as important as the others with regard to attendance of committee meetings within the constitution. This document will explain the issues this creates and the latency it brings to decision making that many members of the committee are not interested in.

With the recent disciplinary processes we've carried out, within the committee and with SUSU, there have been obvious issues with the system currently in place for bringing people through our internal disciplinary process and liasing with SUSU where necessary. This document will put forward changes that along with the restructure of the committee should allow the society to better handle disciplinary matters.

\section{Disciplinary Hearings:}
The existence of Disciplinary hearings is necessitated by the constitutional requirement to handle people that breach the rules of SVGE. People break rules for a variety of reasons so it is of high importance that our internal process is flexible enough to offer pastoral support to the person in question if required, while also taking disciplinary action that a majority within the society agrees with. Our current process does not cater at all to the former of these requirements and recent internal processes have brought this to light in a profound fashion, the details of this are not to be disclosed due to the sensitive nature of the issues.

There is also no condition on which to contact SUSU about the case to escalate it into their hands, for cases where rule breaches have been against University or SUSU's policy it is often important to at least make them aware of the issue and possible allow them to take on the process in place of our internal hearing system.

Disciplinary hearings may currently be held before the Committee of SVGE in private, or before an EGM called for the purpose of handling the hearing. Both of these processes are stressful for all involved and do not take into account the difficulty of public speaking and the privacy concerns that may be related to the hearing in question.

\subsection{Proposed Changes to Disciplinary Hearings:}
Reformation of Disciplinary Hearings such that they are constitutionally required to be held before a three person sub-committee disciplinary panel, chaired by a randomly selected person out of the President, Vice President, Treasurer and Secretary, with two other randomly selected ``Officers of the Committee''. The random selection shall know the identities of the complainants and those accused and ensure that none of them are on the panel. 

If by a majority vote, the the Committee deem there to be a conflict of interest for an ``Officer of the Committee'' who would otherwise sit on the three person panel, that ``Officer of the Committee'' would be excluded from sitting on said panel and hearing the disciplinary case. A random selection would be made of the remaining members. If an ``Officer of the Committee'' feels they have a conflict of interest or would not be able to fulfil the role, they may step down and a new member would be randomly selected.

\subsection{Proposed Changes to Committee}
The committee is currently too large to function properly. Organising around the current 15 member's timetables is incredibly difficult. This slows the activity of the society down. It is therefore proposed that the committee is split into two groups:

\begin{itemize}
    \item Core Committee
    \item Scene Committee
\end{itemize}

The goal of this would be to streamline the every day activities of the society, whilst still enabling scene specific activities to run effectively. These members would be expected to regularly attend and vote in meetings. The core committee would be formed of:
\begin{itemize}
    \item President
    \item Vice-President
    \item Secretary
    \item Treasurer
    \item Competitions Officer
    \item Media and Marketing
    \item Social Secretary
    \item Welfare Secretary
    \item Equipment Officer
    \item Events Officer
    \item Webmaster
\end{itemize}

The scene committee would then be the remaining scene reps. They would be entitled to attend any meeting but would not be voting members, unless at the discretion of the chair. They would instead would be represented by the compeitions officer.

\end{document}
