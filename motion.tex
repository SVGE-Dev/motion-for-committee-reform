\documentclass[10pt, a4paper]{article}

\usepackage{graphicx}
\usepackage{fullpage}
\usepackage{url}
\usepackage{parskip}

\title{Proposals: Restructure of the Committee of SVGE and Reformation of the Disciplinary Process of SVGE}
\author{Frederick Fillingham}

\begin{document}

\maketitle

\section{Introduction:}
The Southampton Video Games and Esports Society (SVGE) currently has 19 committee roles, all of these roles are deemed as important as the others with regard to attendance of committee meetings within the constitution. This document will explain the issues this creates and the latency it brings to decision making that many members of the committee are not interested in.

With the recent disciplinary processes we've carried out, within the committee and with SUSU, there have been obvious issues with the system currently in place for bringing people through our internal disciplinary process and liasing with SUSU where necessary. This document will put forward changes that along with the restructure of the committee should allow the society to better handle disciplinary matters.

\section{Disciplinary Hearings:}
The existence of Disciplinary hearings is necessitated by the constitutional requirement to handle people that breach the rules of SVGE. People break rules for a variety of reasons so it is of high importance that our internal process is flexible enough to offer pastoral support to the person in question if required, while also taking disciplinary action that a majority within the society agrees with. Our current process does not cater at all to the former of these requirements and recent internal processes have brought this to light in a profound fashion, the details of this are not to be disclosed due to the sensitive nature of the issues.

There is also no condition on which to contact SUSU about the case to escalate it into their hands, for cases where rule breaches have been against University or SUSU's policy it is often important to at least make them aware of the issue and possible allow them to take on the process in place of our internal hearing system.

Disciplinary hearings may currently be held before the Committee of SVGE in private, or before an EGM called for the purpose of handling the hearing. Both of these processes are stressful for all involved and do not take into account the difficulty of public speaking and the privacy concerns that may be related to the hearing in question.

\subsection{Proposed Changes:}
Reformation of Disciplinary Hearings such that they are constitutionally required to be held before a three person sub-committee disciplinary panel, chaired by a new committee member: the ``Disciplinary Officer'', with two other ``Officers of the Committee'' present based on availability, seniority and familiarity with the disciplinary process. If by a majority vote, the the Committee deem there to be a conflict of interest for an ``Officer of the Committee'' who would otherwise sit on the three person panel, that ``Officer of the Committee'' would be excluded from sitting on said panel and hearing the disciplinary case.

\end{document}
